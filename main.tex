\documentclass{article}

% TODO:
% - Explain JSON5 features in detail.

% UTF-8 support.
\usepackage[utf8]{inputenc}

% English hyphenation and typographical settings.
\usepackage[english]{babel}

% Stops tildes (~) from being superscripted.
\usepackage[T1]{fontenc}
\usepackage{lmodern}

% Hyperlink support.
\usepackage{hyperref}

% Citations and references.
\usepackage[backend=biber]{biblatex}
\addbibresource{references.bib}

% Indentation support.
\usepackage{changepage}

% Syntax highlighting.
\usepackage{minted}
\usemintedstyle{vs} % Visual Studio

% Stops single quotes from getting too fancy.
\usepackage{upquote}

% Don't hyphenate these words.
\hyphenation{ECMAScript}


% Custom Commands

\newenvironment{gramprod}[2]{
% Formats a grammar production rule.
% Not intended to be used directly.
% Usage: \begin{gramprod}{production-name}{punctuation}
	\begin{samepage}
	\begin{adjustwidth}{0.5cm}{0.5cm}
		\emph{#1} \textbf{#2}
		\begin{adjustwidth}{0.5cm}{0.5cm}
		}{
		\medskip
		\end{adjustwidth}
	\end{adjustwidth}
	\end{samepage}
}

\newenvironment{lexprod}[1]{
% Formats a lexical grammar production rule.
% Usage: \begin{lexprod}{production-name}
% Example:
% \begin{lexprod}{JSON5Number}
% 	\token{JSON5NumericLiteral}\\
% 	\lit{+} \token{JSON5NumericLiteral}\\
% 	\lit{-} \token{JSON5NumericLiteral}
% \end{lexprod}
	\begin{gramprod}{#1}{::}
	}{
	\end{gramprod}
}

\newenvironment{synprod}[1]{
% Formats a syntactic grammar production rule.
% Usage: \begin{synprod}{production-name}
% Example:
% \begin{synprod}{JSON5MemberName}
% 	\token{JSON5Identifier}\\
% 	\token{JSON5String}
% \end{synprod}
	\begin{gramprod}{#1}{:}
	}{
	\end{gramprod}
}

\newcommand{\token}[1]{\emph{#1}}
% Formats a grammar token.
% Usage:   \token{production-name}
% Example: \token{JSON5Identifier}

\newcommand{\lit}[1]{\textbf{\texttt{#1}}}
% Formats a literal grammar token.
% Usage:   \lit{literal-text}
% Example: \lit{Infinity}

\newcommand{\opt}[1]{#1\textsubscript{opt}}
% Formats an optional grammar token.
% Usage:   \opt{token}
% Example: \opt{\lit{,}}

\newcommand{\esref}[2]{\href{http://www.ecma-international.org/ecma-262/5.1/\#sec-#2}{#1}}
% Creates a hyperlink to ES5 spec.
% Usage:   \esref{text}{section}
% Example: \esref{BooleanLiteral}{7.8.2}


% Document

\title{The JSON5 Data Interchange Format}

\author{
	Jordan Tucker
	\thanks{
		In collaboration with Aseem Kishore
		(\href{https://github.com/aseemk}{github.com/aseemk})
	}\\
	\href{https://github.com/jordanbtucker}{github.com/jordanbtucker}
}
\date{March, 2016}

\begin{document}

\maketitle

\begin{abstract}
The JSON5 Data Interchange Format is a proposed extension to JSON that aims to
make it easier for humans to write and maintain by hand. It does this by adding
some minimal syntax features directly from ECMAScript 5.1.
\end{abstract}

\tableofcontents

\section{Introduction}

JSON \cite{json} \cite{rfc7159} is an excellent data interchange format, but it
has limitations that can cause problems when written by hand. The JSON5 Data
Interchange Format (JSON5) is a proposed extension to JSON that aims to
alleviate some of the limitations of JSON by expanding its syntax to include
some productions from ECMAScript 5.1 \cite{es5}.

JSON5 is a subset of ECMAScript 5.1 that adds no new data types, and is
backwards compatible with JSON documents.

Similar to JSON, JSON5 can represent four primitive types (strings, numbers,
booleans, and null) and two structured types (objects and arrays).

A string is a sequence of zero or more Unicode characters \cite{unicode}. Note
that this citation references the latest version of Unicode rather than a
specific release. It is not expected that future changes in the Unicode
specification will impact the syntax of JSON5.

An object is an unordered collection of zero or more name/value pairs, where a
name is a string or identifier and a value is a string, number, boolean, null,
object, or array.

An array is an ordered sequence of zero or more values.

\subsection{Summary of Features}

The following ECMAScript 5.1 features, which are not supported in JSON, have
been extended to JSON5.

\subsubsection*{Objects}

\begin{itemize}
	\item Object keys may be an ECMAScript 5.1 \esref{\token{IdentifierName}}{7.6}.
	\item Objects may have a single trailing comma.
\end{itemize}

\subsubsection*{Arrays}

\begin{itemize}
	\item Arrays may have a single trailing comma.
\end{itemize}

\subsubsection*{Strings}

\begin{itemize}
	\item Strings may be single quoted.
	\item Strings may span multiple lines by escaping new line characters.
	\item Strings may include character escapes.
\end{itemize}

\subsubsection*{Numbers}

\begin{itemize}
	\item Numbers may be hexadecimal (base 16).
	\item Numbers may have a leading or trailing decimal point.
	\item Numbers may be Infinity, -Infinity, NaN, or -NaN \cite{float}.
	\item Numbers may begin with an explicit plus sign.
\end{itemize}

\subsubsection*{Comments}

\begin{itemize}
	\item In-line and block comments are allowed.
\end{itemize}

\subsubsection*{Whitespace}

\begin{itemize}
	\item Additional whitespace characters are allowed.
\end{itemize}

\subsection{Short Example}

The following example illustrates many of the features of JSON5.

\begin{minted}{js}
{
	foo: 'bar',
	while: true,

	this: 'is a \
multi-line string',

	// this is an inline comment
	here: 'is another', // inline comment

	/* this is a block comment
	   that continues on another line */

	movie: 'The Great Escape\x21',

	hex: 0xDEADbeef,
	half: .5,
	delta: +10,
	to: Infinity,   // and beyond!

	finally: 'a trailing comma',
	oh: [
		"we shouldn't forget",
		'arrays can have',
		'trailing commas too',
	],
}
\end{minted}

\section{Grammar}

JSON5 is defined by a lexical grammar and a syntactic grammar. The lexical
grammar defines productions that translate text into tokens, and the syntactic
grammar defines productions that translate tokens into a JSON5 value.

All productions that do not begin with the characters ``JSON5'' are defined by
productions of the \esref{ECMAScript 5.1 Lexical Grammar}{5.1.2} \cite{es5}.

\subsection{Lexical Grammar}

The lexical grammar for JSON5 has as its terminal symbols characters (Unicode
code units \cite{unicode}) that conform to the rules for
\token{JSON5SourceCharacter}. It defines a set of productions, starting from the
goal symbol \token{JSON5InputElement}, that describe how sequences of such
characters are translated into a sequence of input elements.

Input elements other than whitespace and comments form the terminal symbols for
the syntactic grammar for JSON5 and are called tokens. These tokens are the
identifiers, literals, and punctuators of the JSON5 language. Simple whitespace
and comments are discarded and do not appear in the stream of input elements for
the syntactic grammar.

Productions of the lexical grammar are distinguished by having two colons ``::''
as separating punctuation.

\subsubsection*{Syntax}

\begin{lexprod}{JSON5SourceCharacter}
	\esref{\token{SourceCharacter}}{6}
\end{lexprod}

\begin{lexprod}{JSON5InputElement}
	\token{JSON5WhiteSpace}\\
	\token{JSON5LineTerminator}\\
	\token{JSON5Comment}\\
	\token{JSON5Token}
\end{lexprod}

\begin{lexprod}{JSON5Token}
	\token{JSON5Identifier}\\
	\token{JSON5NullLiteral}\\
	\token{JSON5BooleanLiteral}\\
	\token{JSON5Number}\\
	\token{JSON5String}
\end{lexprod}

\begin{lexprod}{JSON5WhiteSpace}
	\esref{\token{WhiteSpace}}{7.2}
\end{lexprod}

\begin{lexprod}{JSON5LineTerminator}
	\esref{\token{LineTerminator}}{7.3}
\end{lexprod}

\begin{lexprod}{JSON5LineTerminatorSequence}
	\esref{\token{LineTerminatorSequence}}{7.3}
\end{lexprod}

\begin{lexprod}{JSON5Comment}
	\esref{\token{Comment}}{7.4}
\end{lexprod}

\begin{lexprod}{JSON5Identifier}
	\esref{\token{IdentifierName}}{7.6}
\end{lexprod}

\begin{lexprod}{JSON5NullLiteral}
	\esref{\token{NullLiteral}}{7.8.1}
\end{lexprod}

\begin{lexprod}{JSON5BooleanLiteral}
	\esref{\token{BooleanLiteral}}{7.8.2}
\end{lexprod}

\begin{lexprod}{JSON5NumericLiteral}
	\esref{\token{NumericLiteral}}{7.8.3}\\
	\lit{Infinity}\\
	\lit{NaN}
\end{lexprod}

\begin{lexprod}{JSON5Number}
	\token{JSON5NumericLiteral}\\
	\lit{+} \token{JSON5NumericLiteral}\\
	\lit{-} \token{JSON5NumericLiteral}
\end{lexprod}

\begin{lexprod}{JSON5String}
	\esref{\token{StringLiteral}}{7.8.4}
\end{lexprod}

\subsection{Syntactic Grammar}

The syntactic grammar for JSON5 has tokens defined by the lexical grammar as its
terminal symbols. It defines a set of productions, starting from the goal symbol
\token{JSON5Text}, that describe how sequences of tokens can form syntactically
correct JSON5 values.

When a stream of characters is to be parsed as a JSON5 value, it is first
converted to a stream of input elements by repeated application of the lexical
grammar; this stream of input elements is then parsed by a single application of
the syntactic grammar. The program is syntactically in error if the tokens in
the stream of input elements cannot be parsed as a single instance of the goal
nonterminal \token{JSON5Text}, with no tokens left over.

Productions of the syntactic grammar are distinguished by having just one colon ``:''
as punctuation.

\subsubsection*{Syntax}

\begin{synprod}{JSON5Text}
	\token{JSON5Value}
\end{synprod}

\begin{synprod}{JSON5Value}
	\token{JSON5NullLiteral}\\
	\token{JSON5BooleanLiteral}\\
	\token{JSON5Number}\\
	\token{JSON5String}\\
	\token{JSON5Object}\\
	\token{JSON5Array}
\end{synprod}

\begin{synprod}{JSON5Object}
	\lit{\{} \lit{\}}\\
	\lit{\{} \token{JSON5MemberList} \opt{\lit{,}} \lit{\}}
\end{synprod}

\begin{synprod}{JSON5MemberList}
	\token{JSON5Member}\\
	\token{JSON5MemberList} \lit{,} \token{JSON5Member}
\end{synprod}

\begin{synprod}{JSON5Member}
	\token{JSON5MemberName} \lit{:} \token{JSON5Value}
\end{synprod}

\begin{synprod}{JSON5MemberName}
	\token{JSON5Identifier}\\
	\token{JSON5String}
\end{synprod}

\begin{synprod}{JSON5Array}
	\lit{[} \lit{]}\\
	\lit{[} \token{JSON5ElementList} \opt{\lit{,}} \lit{]}
\end{synprod}

\begin{synprod}{JSON5ElementList}
	\token{JSON5Value}\\
	\token{JSON5ElementList} \lit{,} \token{JSON5Value}
\end{synprod}

\printbibliography

\end{document}
